% ##############################################################################
% DOCUMENT AND PDF SETTINGS
% ##############################################################################

% --------------------------------------------------
% ENCODING
% --------------------------------------------------
\usepackage{cmap}           % PDF character encoding
\usepackage[T1]{fontenc}    % 8-bit font encoding
\usepackage[utf8]{inputenc} % UTF-8 input encoding

% --------------------------------------------------
% LANGUAGE
% --------------------------------------------------
% \usepackage[ngerman]{babel} % Uncomment this if you write in German
\usepackage{csquotes}         % Package for better quotations
\hyphenation{}                % Here you can define custom hyphenations

% This package can usually be deleted as this is only used to display some
% blindtext.
\usepackage{blindtext}

% --------------------------------------------------
% DOCUMENT VARIABLES
% --------------------------------------------------
% Personal data
\newcommand{\docAuthor}{YOUR NAME}
\newcommand{\docMatriculationNumber}{YOUR MATRICULATION NUMBER}
\newcommand{\docStudyProgram}{YOUR STUDY PROGRAM}
\newcommand{\docStudyFaculty}{YOUR STUDY FACILITY}
\newcommand{\docStudyDegree}{YOUR STUDY DEGREE}
\newcommand{\docStreetName}{YOUR STREET NAME}
\newcommand{\docPostalCode}{YOUR POSTAL CODE}
\newcommand{\docCity}{YOUR CITY}
\newcommand{\docEmail}{YOUR E-MAIL}

% Document data
\newcommand{\docTitle}{This is the title of this thesis} % This is the main title of this document (thesis)
\newcommand{\docSubTitle}{This is a subtitle}            % leave empty to remove
\newcommand{\docType}{Thesis}                            % Usually this will be 'Thesis"
\newcommand{\docSupervisor}{YOUR SUPERVISOR}             % Your supervisor
\newcommand{\docCoSupervisor}{YOUR CO-SUPERVISOR}        % Your co-supervisor
\newcommand{\docDeadline}{DEADLINE}                      % This is the date when you submit the document

% --------------------------------------------------
% PDF SETTINGS
% --------------------------------------------------
\usepackage[
  colorlinks=true,
  linkcolor=black,
  citecolor=black,
  filecolor=black,
  urlcolor=black,
  bookmarks=true,
  bookmarksopen=true,
  bookmarksopenlevel=3,
  bookmarksnumbered,
  plainpages=false,
  pdfpagelabels=true,
  hyperfootnotes,
  pdftitle ={\docTitle},
  pdfauthor={\docAuthor},
  pdfcreator={\docAuthor}
]{hyperref}

% ##############################################################################
% FONTS, SPACING AND LAYOUT
% ##############################################################################
\renewcommand{\baselinestretch}{1.5} % default 1.5 spacing
\raggedbottom                        % don't stretch spacing to fit page length
\usepackage{lmodern}                 % Allow font sizes at arbitrary sizes

% --------------------------------------------------
% PAGE LAYOUT
% --------------------------------------------------
% Page layout, see http://mirrors.ctan.org/macros/latex/contrib/geometry/geometry.pdf #3 Page geometry
\usepackage[
  bindingoffset=1.5cm, % Specify an offset if you plan to bind the document
  inner=2.5cm,         % Left Spacing
  outer=2.5cm,         % Right spacing
  top=3cm,             % Top spacing
  bottom=2cm,          % Bottom spacing
  twoside              % Delete this if you don't want to use double-sided pages
]{geometry}

% --------------------------------------------------
% PAGE HEADER
% --------------------------------------------------
\usepackage[
  headsepline,     % seperator line beneath page header on normal pages
  plainheadsepline % seperator line beneath page header on pages like ToC
]{scrlayer-scrpage}

\clearpairofpagestyles                  % clear default settings
\addtokomafont{pagehead}{\normalfont}   % use normal font for page header
\ohead*{\thepage}                       % page number
\ihead*{\leftmark}                      % chapter name
\setlength{\headheight}{18pt}           % set headheight to 18pt

% --------------------------------------------------
% PAGE FOOTER
% --------------------------------------------------
\addtolength{\skip\footins}{0.5cm}      % spacing at the end of page (footnotes)
\usepackage[bottom]{footmisc}           % Always place footer at end of page

% --------------------------------------------------
% LAYOUT PACKAGES
% --------------------------------------------------
\usepackage{multicol}                   % Format text in multiple columns
\usepackage{emptypage}                  % Support for empty pages via \cleardoublepage
\usepackage{lscape}                     % Support for landscape pages

% ##############################################################################
% UTILITY PACKAGES
% ##############################################################################
\usepackage{environ} % Define custom environments

% ##############################################################################
% BIBLIOGRAPHY
% ##############################################################################
% Support referencing external sources using the biblatex format.
\usepackage[urldate=comp]{biblatex}

% Sources can be added by specifying the below command one or more times.
\addbibresource{bibliography/test.bib}

\usepackage[acronym, toc, nogroupskip, nopostdot, nonumberlist]{glossaries}
\makeglossaries
% \input{bibliography/acronyms}

% ##############################################################################
% IMAGES, FIGURES, DIAGRAMS AND TABLES
% ##############################################################################
\usepackage{graphicx}                   % Support for including images
\graphicspath{{img/}}                   % Default path

% Simple numbering without chapter
\renewcommand{\thefigure}{\arabic{figure}}

% --------------------------------------------------
% TABLES
% --------------------------------------------------
\usepackage{tabularray}           % Improves tabular and longtable
\UseTblrLibrary{varwidth,counter} % Enable support for varwidth (better width measurement) & counters (counters don't step multiple times)

% Defines a new custom table environment named 'longtbl'. It has support these
% parameters: Optional (default empty) custom options for tabularray's longtblr,
% the colspec and the caption.
\NewEnviron{longtbl}[3][]{
  \begin{longtblr}[caption={#3},expand=\BODY]{
      colspec = {#2},
      width = \linewidth,
      measure = vbox,
      hline{1} = {0.4pt},
      hline{2} = {1pt},
      hline{3-Z} = {0.4pt},
      row{1} = {font = \bfseries},
      row{2-Z} = {font = \ttfamily},
      #1
    }
    \BODY
  \end{longtblr}
}

% --------------------------------------------------
% TIKZ DIAGRAMS
% --------------------------------------------------
\usepackage{tikz}      % Draw TikZ figures
\usepackage{float}     % Floating objects
\usepackage{bytefield} % Draw network protocal packets

\usetikzlibrary{shapes.geometric, arrows, positioning, decorations.pathreplacing, calc}
\tikzstyle{box} = [rectangle, minimum width=3cm, minimum height=1cm, text centered, draw=black, fill=orange!30]
\tikzstyle{mainbox} = [rectangle, minimum width=3cm, minimum height=1cm, text centered, draw=black, fill=green!30]
\tikzstyle{plainbox} = [rectangle, minimum width=3cm, minimum height=1cm, text centered, draw=black, fill=white,thick]
\tikzstyle{border} = [rectangle, minimum width=3cm, minimum height=1cm, text centered, draw=black]
\tikzstyle{arrow} = [thick,->,>=stealth]

\pgfdeclarelayer{foreground}
\pgfdeclarelayer{background}
\pgfsetlayers{background,main,foreground}

% Automatically include TikZ figures from external .tix files, setting the
% caption and label.
\newcommand{\tikzfigure}[3]{%
  \vspace{\baselineskip}%
  \begin{figure}[H]%
    \centering%
    \input{#1.tikz}%
    \caption{#2}%
    \label{fig:#3}%
  \end{figure}%
}

% ##############################################################################
% CODE LISTINGS
% ##############################################################################
\usepackage{listings}
\usepackage{color}
\usepackage{xcolor}

\definecolor{backcolour}{rgb}{0.95,0.95,0.95} % Custom color for code background
\definecolor{codegreen}{rgb}{0,0.6,0}         % Custom color for comments in your code
\definecolor{codegray}{rgb}{0.5,0.5,0.5}      % Custom color for numbers in your code
\definecolor{codepurple}{rgb}{0.58,0,0.82}    % Currently unused

\renewcommand{\lstlistingname}{Code Snippet} % Here you can change the code listing name

\lstdefinestyle{normal}{
  backgroundcolor=\color{backcolour},
  commentstyle=\color{codegreen},
  keywordstyle=\color{magenta},
  numberstyle=\tiny\color{codegray},
  stringstyle=\color{orange},
  basicstyle=\ttfamily\small,
  breakatwhitespace=false,
  breaklines=true,
  captionpos=b,
  keepspaces=true,
  numbers=none,
  numbersep=5pt,
  showspaces=false,
  showstringspaces=false,
  showtabs=false,
  tabsize=2,
  rulecolor=\color{black},
  frame=L,
  xleftmargin=6px
}
\lstset{style=normal}

% ##############################################################################
% MISCELLANEOUS
% ##############################################################################

% --------------------------------------------------
% SUPPORT FOR TEXT IN MATH EXPRESSIONS
% --------------------------------------------------
\usepackage{amstext}

% --------------------------------------------------
% SUPPORT FOR DIRECTORY TREE RENDERING
% --------------------------------------------------
\usepackage{dirtree}